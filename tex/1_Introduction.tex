%!TEX encoding = IsoLatin

%
% Chapitre "Introduction"
%

\chapter{Introduction}
\label{s:intro}

Avec les avancements technologiques des dernières décennies, l'accès à la donnée devient un besoin de plus en plus grandissant. Avoir sous la main des statistiques précises dans un certain secteur d'activité rend la tâche grandement plus facile dans l'optimisation d'un produit ou d'un service pour les firmes d'ingénierie. Avec ce nouvel accès à l'information, il est maintenant possible de cibler avec exactitude les besoins d'un client, multiplier la vitesse de production d'un service et même rendre des procédés complètement automatisés.

Dans le projet Fish \& Chips, il sera justement question de développer un design conceptuel d'un capteur permettant la documentation de la faune aquatique dans un milieu donné.

Le mandat fourni par le ministère de la Faune Aquatique impose donc une identification précise des populations de poissons, une collecte fiable d'images à des fins statistiques ainsi que l'accès à une base de données. Bref, le développement de ce produit pourra se traduire en deux principaux aspects : l'implantation d'un logiciel capable de fournir des données avec une fiabilité et une sécurité accrues, et la création d'un concept de capteur multidisciplinaire qui répond aux standards de qualité du client. 

D'abord, ce rapport présente la description du projet ainsi que les besoins et objectifs recherchés. Puis, il aborde le cahier des charges, la conceptualisation et l’analyse de faisabilité, l’étude préliminaire et le concept retenu de la solution présenté au Ministère de la Faune Aquatique.





